%% ---------------------------------------------------------------
%% $URL: https://repository.cs.ru.is/svn/thesis-template/trunk/ruthesis/latex/DEGREE-NAME-YEAR.tex $
%% $Id: DEGREE-NAME-YEAR.tex 360 2019-02-13 22:04:35Z foley $
%% This is a template LaTeX file for dissertations, theses, or reports at Reykjavík University
%%
%% Comments and questions can be sent to the RU LaTeX group (latex AT list.ru.is)
%% ---------------------------------------------------------------

%% METHOD:
%% 0) Read ruthesis/thesis-instructions.pdf
%%    If it is missing, goto https://repository.cs.ru.is/svn/thesis-template/trunk/ruthesis/thesis-instructions.pdf
%% 0.2) Subscribe to the announcements email list at
%%    https://list.ru.is/mailman/listinfo/latex-announcements
%% 1 LaTeX instructions.tex or goto http://afs.rnd.ru.is/project/thesis-template/trunk/ruthesis/latex/instructions.pdf
%% 2) Copy the template files (or unzip) to your working area
%% 3) Rename this file (if needed) with your information e.g. MSC-FOLEY-2007.tex
%% 4) Modify this file to fit your needs (please follow all comments below in the text)
%% 5) For making bibliographies, run "biber".  You can also change
%%    this back to "bibtex".  See below in "Bibliography options".

%%%%%%% CHOOSE ONE OF THESE %%%%%%%%%%%%%%%
%% projectreport: Project report (CS)
%% bachelors: Bachelor of Science thesis
%% masters: Master of Science thesis
%% doctorate: Doctor of Philosophy dissertation
%
%%%%%%% CHOOSE ONE OF THESE %%%%%%
%%
%% draft: speed up processing by skipping graphics and adding useful
%%     information for editing.  Also sets spacing to double so that it is easier to
%%     write editing marks on paper copy.
%% proof:  proofreading version (final formatting with warnings)
%% final: generate document for submission, removing FIXMEs, and
%%     other markup.  Throw error if any fatal FIXMEs still in document.
%%
%%%%%%% CHOOSE ONE OF THESE IF APPLICABLE %%%%%%
%%
%% deptsse: School of Science and Engineering
%% deptscs: School of Computer Science
%%
%%%%%%%% CHOOSE ANY COMBINATION OF THESE %%%%%%%%%%%%
%%
%% forcegraphics: force graphics, etc. to be included, even in draft mode
%% debug:  writes more messages to the log file, adds debugging output
%%     and sizing boxes
%% icelandic: thesis is in Icelandic
%% oldstyle:  use the PhD headers and footers from the old CS template
%% online: for online versions (skip blank pages)
\documentclass[online,masters,deptscs,forcegraphics,draft]{ruthesis}

%%%%%%%%%%%%%%%%%%%% TeXStudio Magic Comments %%%%%%%%%%%%%%%%%%%%%
%% These comments that start with "!TeX" modify the way TeXStudio works
%% For details see http://texstudio.sourceforge.neit/manual/current/usermanual_en.html   Section 4.10
%%
%% What encoding is the file in?
% !TeX encoding = UTF-8
%% What language should it be spellchecked?
% !TeX spellcheck = en_US
%% What program should I compile this document with?
% !TeX program = xelatex

%%%%%%%%%%%%%%%%%%%% Bibliography options %%%%%%%%%%%%%%%%%%%%%
%% We suggest switching from bibtex to biblatex/biber because it is better able
%% to deal with Icelandic characters and other bibliography issues
%% As long as you use biblatex instead of bibtex by itself, it will at least
%%  generate a document without errors.
%% !!!If you are using TeXStudio, don't forget to update the bibliography setting!!!
\usepackage[backend=bibtex,bibencoding=utf8,style=ieee]{biblatex}
%\DeclareLanguageMapping{american}{american-apa}
% need to declare mapping for style=apa to alphabetize properly
% If you set backend=bibtex, it will use bibtex for processing (old way)
%    this can work with Icelandic characters, but you may get weird results.
%    bibtex does not know how to sort Þ and ð
% if you set backend=biber, you can use UTF8 characters such as Þ and
%     ð  but you will have to remember to switch from using bibtex to
%
%     biber in your client
% If you use JabRef, make sure the file is encoded in UTF-8 which is
%    not the default.

%% This tells TeXStudio to use biber
% !TeX TXS-program:bibliography = txs:///biber
%% This also sets the bibliography program for TeXShop and TeXWorks
% !BIB program = biber

% Where is your reference library?
\addbibresource{references.bib}

%%%%%%%%%%%%%%%%%%% CUSTOMIZATIONS %%%%%%%%%%%%%%%%%%%%%%%%%%%%%
%% It is not recommended that you customize this file nor
%% ruthesis.cls.  Just fill in the necessary fields.  You should put
%% your macros and packages into a separate file so that it is easier
%% to use updates to the template.  The custom.sty file was created
%% for this reason.  We load this much later so that it can overrite
%% any existing settings
\IfFileExists{custom.sty}{\usepackage{custom}}{}


%%%%%%%%%%%%%%% INFORMATION %%%%%%%%%%%%%%%%%%5
%% University information must be multilingual to deal with the
%%  required cover pages and abstract on thesis
%% NOTE: This may not be required for other reports!!!

%% Babel Icelandic macros are setup  on RedHat at
%% /usr/share/texlive/texmf-dist/tex/generic/babel/icelandic.sty
%% /usr/share/texlive/texmf-dist/tex/generic/babel-icelandic/icelandic.ldf


%% Multilingual macros
%\newML{macroname}{englishword}{icelandicword}
%  creates \macronameML
%    \MLmacroname[english] - returns the english word
%    \MLmacroname[icelandic] - returns the icelandic word
%    \MLmacroname  - uses the current language setting
% Some useful ones have already been defined, but can be redefined
%% Predefined: \MLIceland \MLReykjavikUniversity \MLUniversityIceland

%% What institute?  Default is RU.
%\setInstitution{\MLReykjavikUniversity}
% \newML{InstitutionAddress}{Menntavegur 1\\101 Reykjavík, Iceland}
% {Menntavegi 1\\101 Reykjavík, Ísland}
% \setInstitutionAddress{\MLInstitutionAddress}
% \newML{Tel}{Tel.}{Sími}
% \setInstitutionPhone{\MLTel{} +354 599 6200\\
% Fax +354 599 6201}
% \setInstitutionURL{www.ru.is}


%% ONLY SET DEPARTMENT IF YOU HAVE NOT USED THE deptsse or deptscs OPTION!
%% Department and degree program
%\newML{ND}{New Department}{Nytt deild}
%\setSchool{\MLND}

%% Set your program of study
\newML{program}{Computer Science}{Tölvunnarfræði}
\program{\MLprogram}

%% Degree long name.  If not already defined, you can create a macro
%\newML{DEGREE}{English Degree Name}{Icelandic Degree Name}
%% Default is set based upon doctorate vs masters option
%% Predefined: \MLMSc \MLPhd
%\setDegreelong{\MLMSc}

%% Degree abb, change if default is not right
%% Default is set based upon doctorate vs masters option
%\degreeabbrv{Sc.D.}

%\setFrontLogo{reyst-logo}
%% Use this if you need a different front logo on the first page
%% e.g. reyst-logo

%% Date in english and icelandic
%% NOTE: THIS IS THE DATE OF THE SUPERVISOR'S SIGNATURE!!!!!!
%% Predefined: \MLjan, \MLfeb, \MLmar, ... \MLdec
%\whensigned{day}{month}{year} %day is only used on some formats, but you must put something.
\whensigned{17}{\MLjan}{2021}

%% Title first in English then Icelandic
%% You need to put both a normal case and ALL CAPS version into the macros.
%%
\newML{Title}{Strategic Evaluation of Deep Neural Networks in Board Games}{Herkænsku mat á djúptauganetum í leikjum}
\newML{TITLE}{STRATEGIC EVALUATION OF DEEP NEURAL NETWORKS IN BOARD GAMES}{HERKÆNSKU MAT Á DJÚPTAUGANETUM Í LEIKJUM}
%%
\setTitle{\MLTitle}{\MLTITLE}
%% ***** Special Titles ******
%% If the title must be formatted specifically for the cover page or internal pages
%% (typically via line-breaks using the \newline command) then the following commands must be used
%%
%\setTitleCover{\MLTITLE}
%% These two for the internal cover pages, usually not needed
%\newML{TitleInternal}{Internal Title}{Icelandic Internal Title}
%\setTitleInternal{\MLTitleInternal}

%% Author name (should be the same in any language, if not use \newML)
%% If you are writing a Project report with multiple authors, separate them with \\:
%% To keep the names typeset together, you want to use non-breaking spaces: ~
%\author{Firstname1~Lastname1\\Firstname2~Lastname2}
\author{Sigurður Helgason}

%% If the name must be formatted specifically for the signature page
%% (typically via line-breaks) then the following command must be used
%\setAuthorSignature{Student\\Name}
%% This macro adjusts the author name in the headers of the oldstyle formatting
%\setAuthorHeader{StudentLast}

%%% TODO:  Move the bachelor's form separately -- it confuses people. --foley
%%%%%%%%%%%%%%%%%%%%%%%%%% Project Report or Bachelor's Only!!! %%%%%%%%%%%%%%%%%%%%%%%%%%%%%%%%%%%%%%%%%%%
\setCourse{Lokaritgerd}

%%%%%%%%%%%%%%%%%%%%%%%%%% Bachelors Only!!! %%%%%%%%%%%%%%%%%%%%%%%%%%%%%%%%%%%%%%%%%%%
\setID{270695--3459}%kennitala
\setSemester{2021--1}
\setShortSignedDate{1.1.2021}

\setOrganization{}
\setSubProgram{}

%% If the thesis is confidential, uncomment this with the date it can be released
%\setClosedDistribution{10.1.2016}%

%% Put your keywords here in English, then Icelandic.  Separate them with commas.
\newML{keywords}{Model Interpretation, Reinforcement Learning, Deep Learning, Neural Networks, Machine Learning, Game playing, Artificial Intelligence}{Djúptauganet, Tauganet, Gervigreind, Vélrænt Gagnanám}
\setKeywords{\MLkeywords}

%%%%%%%%%%%%%%%%%%%%%%%%%%% Masters Only!! %%%%%%%%%%%%%%%%%%%%%%%%%%%%%%%%%%%%%%%%%%%%
%% How many credits (ECTS) on Master's degree
%% Usually 30 or 60
\ects{60}

%%%%%%%%%%%%%%%%%%%%%%%%%%% Doctorate Only!! %%%%%%%%%%%%%%%%%%%%%%%%%%%%%%%%%%%%%%%%%%
%% Some Computer Science Thesis have an ISSN number.
%% Most other documents do not.
%\bookidnumber{ISSN: 1670-8539}
%% ID numbers are optional, but nice for sorting in libraries

%% International Standard Book Number (ISBN)
%% This is what most people should use if the thesis is being published.

%% International Standard Serial Number (ISSN)
%% This is usually only for a PhD dissertation as part of a series when published
%%   Computer Science: 1670-8539

%% Additional degrees?  (optional, usually not needed)
%\adddegree{(list of degrees in appendix)}{(sjá lista yfir prófgraður í viðauka)}
%%%%%%%%%%%%%%%%%%%%%%%%%%%%%%%%%%%%%%%%%%%%%%%%%%%%%%%%%%%%%%%%%%%%%%%%%%%%%%%%%%%%%%%%


%% List the entire committee.  Each member has a name (degree should be omitted, unless it is not PhD),
%% Supervisor(s) must appear first
%% On a Bachelors, there is usually only one supervisor and one examiner.

%% Format for each entry:
%%  \personinfo{Name}{Role}{Job Title}{Company/institution}{Country}
%% Predefined macros: \MLSupervisor \MLSupervisors \MLExaminer \MLExaminers

%% Change these to singular/plural as needed.
%% Just uncomment and change the plurality of the macro.
%\setSupervisorHeading{\MLSupervisors}
%\setExaminerHeading{\MLExaminer}

%% Predefined macros:
%% \MLSeniorProfessor \MLProfessor \MLAssociateProfessor \MLAdjunctProfessor \MLEmeritusProfessor \Iceland
%% \MLReykjavikUniversity \MLUniversityIceland

%% Bachelors: primary advisor (Umsjónarkennari), ONLY ONE!
%% All others: As many as you want
\supervisors{
  \personinfo{Yngvi Björnsson}{\MLSupervisor}{\MLProfessor}{\MLReykjavikUniversity}{\MLIceland}
%  \personinfo{Helpful A. Teacher}{Co-advisor}{\MLAssistantProfessor}{\MLUniversityIceland}{\MLIceland}
%  \personinfo{Ian M. Great}{Co-advisor}{\MLProfessor}{Hochschule Düsseldorf}{Germany}
}

%% Bachelors: secondary advisor (Leiðbeinandi), ONLY ONE
%% All others: As many as you want
\examiners{
  \personinfo{Stephan Schiffel}{\MLExaminer}{\MLAssistantProfessor}{\MLReykjavikUniversity}{\MLIceland}

  \personinfo{David Thue}{\MLExaminer}{\MLAssistantProfessor}{Carleton University}{Canada}
}

%% An abstract is required to be in both Icelandic and English for most degrees.
%% It is considered good form to limit the abstract to a single paragraph in each language,
%%   at 300 words.  Refer to your degree's instructions.
%% Note: Icelandic quotation marks cannot be typeset using "` and "'.  You should use \enquote{}
%% this is probably due to interactions with the MultiLingual macros.
%% TODO: turn this into more sensible macros to avoid confusion --foley
\newML{AbstractText}{\import{./sections/abstract/}{english}}{\import{./sections/abstract/}{icelandic}} % Icelandic abstract goes here
\setAbstract{\MLAbstractText}


%%%%%%%%%%%%%%INDEX SETUP %%%%%%%%%%%%%%%%%%%%%%%%%%%%%%%%%%%%%%%%%%%%%%%%%%%%
%% Indexes, and other auto-generated material
%% The Memoir package (which we use) automatically generates the index
%% See section 17.2 on page 302 of the guide
%% http://texdoc.net/texmf-dist/doc/latex/memoir/memman.pdf
%% This means you have to run "makeindex DEGREE-NAME-YEAR"
%% !!!Do not load any of the index packages, they cause problems with Memoir!!!
%% !!!You have been warned!!!
%% Note that memoir changes the [] options to only be for filenames, not other options!
\makeindex{}
\indexintoc{}

%% For abbreviations, you may want to try
%% Watch out though, each new index writes another external file and
%% latex can only write a limited number of them
%%\usepackage[intoc]{nomencl} % intoc: In Table of Contents
%% remember to run:
%% makeindex filename.nlo  -s nomencl.ist -o filename.nls

\finalifforcegraphics{hyperref} %hyperlinks even in draft mode
\usepackage[hidelinks]{hyperref}
%% !!!Must be the last package loaded except otherwise mentioned!!!!
%% \usepackage{hypcap}  %% puts link at top of figure, must be after hyperref

%%%%%%%%%%%%%%%%%%%%%%%%%%%%%%%%%%%%%%%%%%%%%%%%%%%%%%%%%%%%%%%%%%%%%%%%%%
%%%%%%%%%%%%%%%%%%%%%%% DOCUMENT START %%%%%%%%%%%%%%%%%%%%%%%%%%%%%%%%%%%
\begin{document}

%% Some elements have different names on the RU Masters rules
%% They will be annotated with RUM: "name"
\frontmatter{} % setup formatting at beginning

%\frontcover{}%%If you want to see what it looks like with the printed cover
%% TODO:  link to fill-in PDF file on RU website

\frontrequiredpages{}%% the various signaturepages and abstract
%%% WARNING:  if you get an error on the previous line, it is probably because
%%% you put a bad macro or something strange in a title, author, or abstract.


%% Dedication is optional, comment out if it is absent
\begin{dedications}
  I dedicate this thesis to my family, who while not understanding most of what I do,
  always support me and the friends that still want to talk to me even though my
  schedule has rendered me unmeetupwithable. Lastly, my girlfriend Hulda, who always helps me get 
  through the tough times, and makes the good times even better.
\end{dedications}

\enableindents{}% turn on/off paragraph indents
\coverchapter{Acknowledgements} 
\begin{quotation}
Throughout the writing of my thesis I always had support and assistance from the people around me.

I would first like to thank my supervisor, Yngvi Björnsson, his knowledge in this field is vast and he is always ready to share invaluable wisdom when it's needed. Your critical feedback and consistent positivity has pushed me and my work to a higher level.

Secondly, I would like to thank my colleagues, who provided stimulating conversations on the topic and a helpful eye from a different perspective every time I needed it.

\end{quotation}
\vspace{\baselineskip}

%\coverchapter{Publications}
%% RUM: Not mentioned, this was found in the CS thesis template.  
%% Maybe more applicable to PhD dissertations?
%%% Probably a duplication from before Preface became standard.

\starttables{}% setup formatting
%% TOC, list of figures and list of tables are required
\tableofcontents{}\clearpage%%RUM: "Table of contents"
\listoffigures{}\clearpage%%RUM: "List of figures"
\listoftables{}\clearpage%%RUM: "List of tables"

%\coverchapter{List of drawings and enclosed material}
%RUM: "List of drawings and enclosed material, e.g. CD(as appropriate)"

\listoffixmes{}
% if using fixme package, lists what needs to be done

%% The list of abbreviations is an example of a special list
%% Other lists may be added, such as lists of algorithms, symbols, theorems, etc.
%% IN CS PhD, this is sometimes centered.
\coverchapter{List of Abbreviations}%%RUM: Not mentioned
\begin{tabular}{ll}
MSc &Masters of Science\\
ML & Machine Learning\\
AI & Artificial Intelligence\\
ANN & Artificial Neural Network\\
DNN & Deep Neural Network\\
MCTS & Monte-Carlo Tree Search\\
CAV & Concept Activation Vector\\
DL & Deep Learning\\
MI & Model Interpretability\\
ME & Model Explainability\\

\end{tabular}
\overfullrule=0pt
%% This command prepares for the actual text, e.g. by
%% calling \mainmatter{}
\starttext{}

%% ---------------------------------------------------------------
%% From this point on, it is standard Latex, except the very end.
%% This is a "report"-based template, so the top-level heading
%% is \chapter{}

%% WARNING: Make sure that all of these files (and any new ones)
%% are UTF-8 otherwise you will get weird encoding errors.
%\part{The First Part} % Parts optional but useful in longer documents


%% The default division is IMRAD, you may want to divide differently
%% See the introduction for guidance.

\chapter{Introduction\label{cha:introduction}}
%% \ifdraft only shows the text in the first argument if you are in draft mode.
%% These directions will disappear in other modes.
\ifdraft{State the objectives of the exercise. Ask yourself:
  \underline{Why} did I design/create the item? What did I aim to
  achieve? What is the problem I am trying to solve?  How is my
  solution interesting or novel?}{}

In the year $2017$ the approximate amount of adults with diabetes was 425-million people \cite{Intro_FactsandFigures}, this number is 
projected to rise to $629$-million people by the year $2045$. The rate of people that are diagnosed with diabetes 
is 1 in every 2. Secondly of all people with diabetes 1 in every 3 develop diabetic retinopaty (DR)\cite{Intro_DiabetesRate}.

\textit{Diabetic retinopathy} is a symptom of both type 1 and type 2 diabetes and is now the leading cause of blindness amongst adults today. DR is 
diagnosed by two major technologies, firstly there is the fundus imaging method, where a photograph is taken of the fundus (back) 
of the patients eye, an example of this can be seen in Figure \ref{fig:fundus-image1}. Secondly ophthalmologists examine 
Optical Choerence Tomography (OCT) scans of human retina, an example of this can be seen in Figure \ref{fig:oct-retina1}. 
The work in this paper does not focus on the latter. The diagnosis of DR is costly and as the statistic indicate 
the amount of work is going to increase.

\begin{figure}[h]
  \centering
  \includegraphics[width=.3\linewidth]{../graphics/fundus_image1.jpg}
  \caption{Example of Fundus image}
  \label{fig:fundus-image1}
\end{figure}
\begin{figure}[h]
  \centering
  \includegraphics[width=.5\linewidth]{../graphics/oct_fundus.png}
  \caption{Example of OCT on Retina}
  \label{fig:oct-retina1}
\end{figure}

Patients that have DR are grouped into four different stages\cite{Intro_FourStages}:

1. \textit{Mild Nonproliferative Retinopathy} - This is the earliest stage of diabetic retinopathy, and it’s characterized by balloon-like swelling in the retina’s blood vessels. These are called microaneurysms, and these vessels can leak into the eye.

2. \textit{Moderate Nonproliferative Retinopathy} - During this stage, the blood vessels nourishing the retina swell and may even become blocked. This can contribute to diabetic macular edema (DME) which is a build-up of fluid in the macula region of the retina.

3. \textit{Severe Nonproliferative Retinopathy} -  At this stage, an increasing number of blood vessels nourishing the eye have become blocked. As a result, the retina is signaled to grow new blood vessels.

4. \textit{Proliferative Diabetic Retinopathy} - This is the final stage of diabetic retinopathy. New blood vessels proliferate, growing inside the retina and into the vitreous gel, which is the fluid that fills the eye. Because these blood vessels are delicate, they may begin to leak and bleed. As a result, scar tissue may form, causing retinal detachment, the pulling away of the retina from underlying tissue. Retinal detachment may cause spotty vision, flashes of lights, or severe vision loss.

Generally without intervention patients advance from one stage to the next eventually leading 
to permanent vision impairment, however with intervention by an ophthalmologist this advancement 
can be prevented \cite{Intro_StageAdvancement}. 

This paper focuses on applying machine learning methods to patient information and Fundus images of those 
patients to assess the likelyhood (risk) of those patients DR developing into SDR.

In this paper we will apply machine learning methods to patients information and fundus images of patients 
that has been collected a period of time to examine a patiens likelyhood to advance from one stage 
of DR to the next, attempting to predict the timeline the patient has to seek medical help. 

\section{Machine learning methods in Ophthalmology}

Applying machine learning to fundus images and patient information has been done before very successfully 
researchers\cite{Intro_googleDeepmind}, this work generally focuses on applying image recognition on 
the images to identify which stage of DR the eye has reached. 

\section{Background}
\ifdraft{Provide background about the subject matter (e.g. How was morse code
developed?  How is it used today?). 
This is a place where there are usually many citations.
It is suspicious when there is not. 
Include the purpose of the different equipment and your design intent. 
Include references to relevant scientific/technical work and books.}

Applying statistics to medical data in order to better understand a patients evolution from DR to SDR 
has been done previously by a company called RetinaRisk, a significant contributor to this paper, but 
RetinaRisk's solution does not apply machine learning but a specialists field knowledge and a handcrafted 
formula. RetinaRisks formula has been used to significantly reduce costs to health organizations as 
when patients following their recommendations for doctor checkups come in less often but at time that 
are more critical. im just writing something so that it looks like I'm writing something
%%RUM: Introduction
\chapter{Background\label{cha:background}}

In this chapter, we discuss traditional artificial intelligence in the context of search, the algorithms we used, what some other similar methods exist as well as the field in general. We allocate a large portion of the section to Breakthrough as that will be the testbed for future sections. We discuss the different classes of machine learning as well as neural networks in more depth.

\section{Environments}
\label{sec:environments}

When researching Artificial Intelligence it is important to select an environment that is a suitable abstraction for the task at hand. Environments vary significantly and can be identified by their characteristic. The characteristics that are generally used to describe environments can be seen in Table~\ref{tab:env_characteristics}. This categorization of environments is described in the book Artificial Intelligence by Norvig \& Russell. \cite{Russell:AIModern}

\begin{table}[ht]
  \centering
  \begin{tabular}{|c|c|p{6cm}|}
    \hline
    \textbf{characteristic} & \textbf{Values}              & \textbf{Description}                                                                 \\
    \hline
    Observable              & Fully, Partially             & How much of the environment can your agent percieve.                                 \\
    \hline
    Agents                  & Single, Multi                & Are there multiple agents playing in the environment.                                \\
    \hline
    Deterministic           & Deterministic, Stochastic    & Do the actions your agent do deterministicly impact the environment.                 \\
    \hline
    Episodic                & Sequential, Episodic         & Are actions episodic or sequential.                                                  \\
    \hline
    Static                  & Static, Semi-Static, Dynamic & Does the environment without agent input, or does it wait until agents take actions. \\
    \hline
    Discrete                & Discrete, Continuous         & Is your environment discrete w.r.t actions.                                          \\
    \hline
  \end{tabular}
  \caption{Characteristics of environments}
  \label{tab:env_characteristics}
\end{table}

Categorizing environments like this gives you the power to find an environment in which a method works and know it can be applied to different environments with the same characteristics. Additionally, it allowes us to talk about agents in the context of environments as the entities that act within the environment.

\section{Game Environment}

Classical Artificial Intelligence Game Environments are commonly used to validate a method, e.g game environments can be games like Tic-Tac-Toe, Breakthrough, and driving simulators. Game environments are a suitable place to apply AI as they serve as an abstraction of the real world, for instance, a self-driving car agent who is verified to avoid driving into walls in a simulation is possibly safer than one who is not.

\section{Breakthrough}

The game Breakthrough is a simplified version of chess; the game is set up on a $MxN$ board with squares like in chess, and each player starts with two rows of pawns at opposite ends. The objective of the game is for a player to move one of their pawns to the opposite end of the board. A player wins if either they have reached the opposite end of the board or have captured all of his opponents' pawns. The pawns differ from chess pawns in such a way that they can not move two squares on the first move and, they can move diagonally as well as forward. This leads to the game being impossible to draw as pieces are always able to move. An example of an initial board in Breakthrough can be seen in Figure~\ref{fig:initbtboard}

\begin{figure}[]
  \centering

  \breakthrough{8/8/pppppp2/pppppp2/8/8/PPPPPP2/PPPPPP2 w - - 0 1}

  \caption{Initial breakthrough board}
  \label{fig:initbtboard}
\end{figure}


\begin{table}[ht]
  \centering
  \begin{tabular}{|c|c|}\hline
    \textbf{Characteristic} & \textbf{Value} \\\hline
    Observable              & Fully          \\
    Agents                  & Multi          \\
    Deterministic           & Deterministic  \\
    Episodic                & Sequential     \\
    Static                  & Static         \\
    Discrete                & Discrete       \\\hline
  \end{tabular}
  \caption{Categorization of Breakthrough}
  \label{tab:breakthrough_cat}
\end{table}

Categorizing Breakthrough with the characteristics described in Section~\ref{sec:environments}. We end up with the description shown in Table~\ref{tab:breakthrough_cat}. These characteristics are identical to that of Tic-Tac-Toe, and chess. This categorization is the most common in board games where two-player compete.

\subsection{Heuristics of Breakthrough}

To evaluate the game of Breakthrough we can consider many heuristics (higher-level concepts) for instance a very simple heuristic would be a players material advantage. \textit{Material Advantage} is the amount of pieces the player has minus the amount of pieces the opponent has. This heuristics gives us some insight into how well the game is progressing, but obviously, there are cases where this doesn't tell us much, as in cases when your opponent has a single piece left that is on the row immediately before the row needed for him to win. No matter how many pieces you have left, this state is bad for you if you're not able to capture that piece. An example of such a state can be seen in Figure~\ref{fig:bt_h1_bad}.

\begin{figure}[]
  \centering
  \breakthrough{8/8/pppp4/pp3P2/2p5/8/8/8 w - - 0 1}
  \caption{Breakthrough board where Material Advantage doesn't work well}
  \label{fig:bt_h1_bad}
\end{figure}

A different heuristic would be the distance of your most advanced pawn minus your opponent's most advanced pawn; this heuristic could give you insight into how close you are to winning the game or how close your opponent is. As a higher-level concept, we can call this concept your aggressiveness, as it closely resembles how aggressive you are going for the win. Generally, in Breakthrough, it is favorable to move your whole team as a unit and play more defensively. Additional heuristics for Breakthrough will be discussed in detail later in this research.

\section{State-Space Search}

Traditionally, methods for playing games search through the environment using a heuristic to guide the search. A simple way of doing a heuristic-based search would be to give all non-terminal states a $0$ score and terminal states positive or negative scores based on whether it is a inw or a loss, repesctively. We say that that a search algorithm is not guided, and the algorithm will probably have to evaluate a large portion of the state-space. This method of searching is generally extremely inefficient as the state-spaces of game environments are often extremely large, even infinite. For instance, an upper-bound estimate of the state-space for Breakthrough is $3^{(M-1)*(N-1)}+2*N$ where $M$ is the height of the board $N$ is the width of the board. The $3^{(M-1)*(N-1)}$ represents each position of the board having either a white, or black piece, or being empty. And, the $2*N$ component represents each square the final piece to move could have moved to. So for a small board, $5x4$ the upper-bound estimation of the state-space is $531,449$ states.

This is why a good heuristic is very valuable because we can disregard all following states that result from doing a move some in a previous state as they will only lead to worse outcomes.

The algorithms that are used in traditional state-space searches are for instance Depth-first search (DFS), Breadth-first Search (BFS), Alpha-Beta Pruning Search (AB-Search)\cite{abpruning:dj}, and Monte-Carlo Tree Search (MCTS).

More modernly, these search methods have been amplified by Machine Learning, in such a way that we do not need to figure out a good heuristic for a given state, but rather, we apply a machine learning model to learn a function that takes in a state and returns an evaluation of that state.\cite{neuralnetworksgames:michulke} This can lead to a significant time reduction as we do not need to simulate a whole game from a state to receive its evaluation we rather receive the evaluation from the model.

\subsection{Monte-Carlo Tree Search}

\label{sec:mcts}

In the algorithm Monte-Carlo Tree Search described by R. Coulom\cite{mcts:coulom}, there is an agent within some environment. Where each node in the environment represents a state-action pair of the environment, by state-action pair what is meant it is some state and the action that brought the agent to that state. This pair should be unique within the environment.

MCTS is a method of exploring an environment in a randomized manner (Monte Carlo is the term implying randomness). In MCTS there are four stages. Selection, Expansion, Simulation, and Back-propagation. They happen sequentially and repeatedly. MCTS is initialized with a tree consisting of the unexpanded initial state of the environment.

In MCTS there is a tree representing the game-environment. This tree consists of nodes $n_i$ where $i$ represents the point in time of the node, for example, $N_0$ in chess is the initial position and $N_x$ is some position in the middle of the game and $N_e$ is one of the states representing a position where there is either a draw or one player has won the match. Each of the nodes has $4$ values, $s$, $a$, $Q$, and $N$. These values represent these items, $s$ is the state of the environment, $a$ is the action that brought the previous node $n_{i-1}$ to node $n_i$, $Q$ is the average reward from running the MCTS algorithm from this node, and $N$ the number of times the MCTS algorithm has visited this node. The values $s$ and $a$ uniquely identify a position in the environment and are often called state-action pairs.

The MCTS algorithms four phases
\begin{enumerate}
  \item Selection
  \item Expansion
  \item Simulation/Rollout
  \item Backpropagation
\end{enumerate}


\begin{equation} \label{UCT_formula}
  \text{Child UCT value} = \frac{Q_{(s',a')}}{N_{(s',a')}} + c_{uct} * \frac{\sqrt{log(N_{(s,a)})}}{N_{(s',a')}}
\end{equation}

\subsection*{Selection}
During the selection phase, a node $(s,a)$ within the tree which has not yet been expanded is found.
This process uses Upper Confidence Bound on Trees (UCT) to find that node $(s,a)$, the formula is described in Equation \ref{UCT_formula}. For a parent node $(s,a)$
(initially the root of the tree) we select the child with the highest UCT value. Repeatedly until an unexpanded
node is found. This process is done to balance the amount of exploration vs exploitation of nodes in the
tree.

\subsection*{Expansion}
Then the expansion phase expands the node generating all of $(s,a)$'s children, $(s',a')$ are generated by applying all actions $a'$ in $(s,a)$.

\subsection*{Rollout}
Next during rollout, actions from $(s,a)$ are randomly selected to move to $(s',a')$, then repeated to go to $(s'',a'')$, until a terminal node within the environment is reached. By terminal, we mean a state in which the game is finished. A terminal node in MCTS can generally return any value, but in the context of this paper, we only return (+1 white wins, or -1 black wins).


\subsection*{Backpropagation}
The result from the terminal node is then propagated up through the path taken by selection $(s,a)$ up
to the root of the tree, updating the $Q_{(s,a)}$ values of each node $(s,a)$.

When training a neural network the UCT formula is modified slightly to prefer selecting nodes
that the neural network values highly by introducing a second scalar to the formula $f((s,a)) = (p,v)$, where $f$ is the neural network, $p$ is the policy vector returned by the neural network and $v$ is the predicted value from the neural network. The resulting formula is described in Equation \ref{PUCT_formula}, and is called PUCT. Secondly, the backpropagation process
is modified to instead of doing rollout/simulation to receive a reward the predicted value $v$ from the neural network is used instead.

\begin{equation} \label{PUCT_formula}
  \text{Child PUCT value} = \frac{Q_{(s',a')}}{N_{(s',a')}} + c_{uct} * p_{(s,a)} * \frac{\sqrt{log(N_{(s,a)})}}{N_{(s',a')}}
\end{equation}

\section{Machine Learning}

Machine Learning (ML) is a research field in which machines apply statistical functions on data to achieve a correct output, by \textit{correct} we mean the corresponding result which we expect. Generally, this a repetitive process where we look at examples of the data, and the algorithm progressively gets closer to the underlying function of the data it is fed. This process is therefore similar to trial and error for humans. ML is a sub-field of Artificial Intelligence. ML algorithms try to achieve one of two classes, \textit{Classification}, where the algorithm should find a class representing the data it is given, and \textit{Regression} where the algorithm should find an underlying continuous numerical function and will return a numerical value representing the input.

Typically ML can be viewed in three different groups, Supervised Learning, Reinforcement Learning (RL), and Unsupervised Learning. Where in Supervised Learning, the algorithm is given data examples and their corresponding outcome. For example, a supervised learning algorithm could be provided with data regarding the weather and the corresponding temperature, the algorithm should then find a pattern within the weather data and find the continuous function represented by the data. This would then be an example of a regression task. Flipping thing example around, if the algorithm would just be provided the temperature and it should tell us whether it is sunny outside or not, that would be a classification task.

% In RL the algorithm is given only some input, and then the algorithm tries some outcome generally actions in some environment. Then over an episode~\footnote{a series of outcomes / a timespan} once the episode is finished some reward is given. The algorithm will then learn whether the actions were good actions from the reward. Examples of this are agents playing a game like Flappy Bird, where the data they are given is the state of the game, and they try to either jump or not jump.

%Machine learning (ML) focuses on the using of data and a corresponding outcome w.r.t that data to 
%extrapolate some underlying function of that data. Examples of how we use machine learning is for instance the ability to predicting the 
%rise and fall of some stock, predicting what the weather will be in a week, and whether an image 
%is an image of a dog or a cat. Many data structures and algorithms are used to achieve this 
%goal, but recently the field of ANN/DNN's has been the standard for achieving the best
%results.

%As a general notion we can split machine learning into three different sects those are Supervised Learning, Reinforcement Learning, and Unsupervised Learning.

\subsection{Supervised Learning}

In Supervised Learning, the ML algorithms attempt to build a model from a data set of labelled examples. The labelled examples are a set of input values and their corresponding output value. The ML algorithm then uses this data set to construct a model that is as accurate as possible at outputting the correct output value given the input value. In supervised learning many techniques are applied to maintain the generality of the model s.t. it does not just represent the data set it is given but also has high accuracy on a possible future data set it has not yet encountered.

Some examples of algorithms that are popular for Supervised learning would be, Decision Trees, Support Vector Machines, or Naive Bayes.

\subsection{Reinforcement Learning}

Reinforcement Learning focuses on the idea of trial and error for an ML algorithm, where the algorithm directly interacts with the environment it operates in, and from operating in the environment it is provided with either positive or negative feedback for it's actions. Typically, the environment needs to be modeled as a Markov Decision Process. That is, the selection of actions in a state requires only the knowledge of being in that state, not the actions it took to get to that state.

Many algorithms are popular in reinforcement learning, for instance Q-learning\cite{qlearning:watkins}, and many others.

\subsection{Unsupervised Learning}

In Unsupervised Learning, the machine learning algorithms attempt to build a model from a data set of values that don't have a corresponding output value. These algorithms then generally attempt to find pattern within the data set, to which we could then later label upon examination of the patterns. Importantly, in Unsupervised Learning, all columns of values in the data set should be normalized to the same range, and should be standardized s.t. the mean of the values is $0$ and it's standard deviation is $1$. This is done in order for one value not to dominate the patterns in the data set.

Common algorithms in Unsupervised Learning are, K-Nearest Neighbours (KNN), K-Means\cite{lloyd:kmeans}, and DBScan\cite{ling:dbscan}.

\subsection{Neural Networks}

Neural networks (NN) are popular methods within a sub-field of ML which is called Deep Learning (DL).
NN's are created to resemble how the human brain functions. In the brain, we have neurons which when they get a signal they apply some function to them and if the resulting signal is high enough, they fire to the next neuron. This is how it is done in the neural network model as well.
There we have neurons that when they get some input, generally a vector of numbers. The neuron takes the sum of that vector, weighs the sum by a constant, then applies an activation function to it. The result of doing this is then passed on to the next neuron. Until a final layer of neurons is reached. At that point, we have a value that the neural network corresponds to the input value. This value can be a binary classification (cat or dog image), a regression value (the value of a property), or any number of outputs. It can then be said that a neural network is doing a function approximation of the input to some value. And, would be mathematically stated as $f_n ( w_n * f_{n-1} (w_{n-1} * \dots f_0(w_0*i))) = o$.

\section{Explainable Artificial Intelligence (XAI)}

The field of XAI research is still very far behind its counterpart AI research, within XAI two fields are the largest, those are Model Interpretability and Model Explainability.

\subsection{Model Interpretability}

Model Interpretability is the more common approach of XAI, mainly because it is less constrained than model explainability. In order to achieve model interpretability, we must be able to answer the question of what prediction will the model return on a given input, with a high accuracy. Simple examples would be, given a trained neural network and a picture of a dog, if that model is highly interpretable, we can say with high certainty that the predicted value will be dog.

\subsection{Model Explainability}

Model explainability within the context of neural networks isn't possible today. Model
explainability referrs to firstly considering some input and output from a model. Then
afterwards the model is examined to determine exactly what led to the predicted output.
This concept is simple when we're working with Decision Trees. A decision tree is a tree
whose nodes are representative of an input value and at every node a branch is selected
based on the value of the input value. It is therefore easy to see how to examine the tree
to explain the output, by following the branches in the tree we can exactly explain why the model predicted the output.

When we talk about neural networks this process is much more difficult, the underlying nodes
are generally in the millions, the different layers of the neural network vary in the operations they apply to the input. During this process, the value is modified such that becomes far removed from the initial input value. That being said, while the possibility of completely monitoring the training process and completely monitoring the evaluation process is truly possible it is not feasible.


\subsection{Saliency Maps}

Within XAI many methods have been developed to try to evaluate
ANN's. In the field image recognition there has been a lot of work examining which
pixels of an image the model deems important. One such method is, Saliency Maps\cite{Koch:saliency}. There the pixel values the model
deems important are colored in s.t. a human can examine the image and get a
sense of what portions of the image are important to the model, an example of
a classification of a dog can be seen in Figure \ref{fig:dog_saliency}. This method is understandable to a human when the saliency map lines up to what we would focus on. However, this achieves no explanation on the prediction if the saliency map doesn't line up to human intuition or the prediction, for example, if in the same Figure\ref{fig:dog_saliency} we had the same saliency map but the prediction would be dragon, or if the salience map was on the trees and the prediction was dog.

\begin{figure}[]
  \centering
  \includegraphics[width=.5\textwidth]{graphics/dog_saliency}
  \caption{Example of a models saliency map for an image of a dog}
  \label{fig:dog_saliency}
\end{figure}

\subsection{Shapley Values}

Methods for explaining models that aren't image recognition models include
Shapley values, from the Lundberg \& Lee\cite{LundbergL:shapley}. There the input is examined against it's output, then iteratively
input values are selected to be fixed. Then the other input values are varied and
an average change in prediction is calculated. With this the shapley value can be
estimated for the fixed input value. This is done to examine which input values have
the strongest link to the output value. Shapley values on a dataset can give insights
on which input values the model deems important.

\subsection{Concept Activation Vectors}

A recent paper by Been Kim et al.\cite{Keem:TCAV}, shows a method for examining
a neural network giving a much more human insight into a prediction. Using Concept
Activation Vectors (CAV) a directional derivative for a given input can be examined
with respect to some HLC's. For example, when a human looks at an image of an animal
and is supposed to decide whether the image is of a horse or a zebra, an intuitive
approach would be to check whether the animal has stipes, or the animal has both white and black colors.
That method of determining if a horse is a zebra could then be called a higher-level
concept, and if we're able to gather if a nerual network uses this strategy for prediction
we have a deeper understanding of its underlying structure. Leading to an explanation of
the result.

The construction of a CAV requires a method of labelling the values in your dataset with
the corresponding concept in order to create a binary classifier on data. The binary classifier is constructed on the internal representation of the data points within the neural network. After training the neural network, and constructing the classifier, when we run a new datapoint through the neural network, and we examine the directional derivative of that datapoint. If the direction is in the direction of the binary classifier we say that the datapoint contains the concept.%%RUM: Background
\chapter{Methods}

\section{Monte Carlo Tree Search}

To train the models  

%\section{Machine learning methods}

This section discusses the various machine learning methods utilized throughout this project 
and discusses their applicability.

%\subsection{Deep neural network}

I'm a little teapot

%\subsection{Data joined network}

I'm a smaller teapot
%%RUM: "Methods"
\chapter{Results}

\ifdraft{In this section you discuss any issues that came up while developing
the system.  If you found something particularly interesting,
difficult, or an important learning experience, put it here.  This is
also a good place to put additional figures and data.}

In this section we discuss the results of considering HLC within the game Breakthrough, using a Concept Activation vector for evaluating how the neural network recognizes the HLC's. We first take a look at the changes in ephasis of the neural network during training. The main point of interest there being whether the nerual network notices simple HLCs early then stops taking them into consideration as the network improves.

\section{Evaluating the improvements of the neural network over generations}

To test which HLC the neural network places its emphasis on during training we trained a neural network for $200$ iterations, taking snapshots of the network every $10$ generations. Then we had the neural network play against itself for $100$ games collecting the states it encountered during play. These states were then examined by a concept activation vector representing these HLCs. Firstly there is the numbers HLC, where the number of pawn the player has minus the number of pawns the opponent had, the breakpoint we selected was $2$ meaning that if you have $2$ more pawns then your opponent, you're in a position where a HLC called numbers advantage is present.

\begin{figure}[]
    \centering
    \includegraphics[width=0.7\textwidth]{graphics/number_pawns_trend}
    \caption{Percentage of selected states containing the HLC numbers advantage}
    \label{fig:numberadvantage}
\end{figure}

The Figure \ref{fig:numberadvantage}. shows that over the course of training the numbers advantage HLC is only ever a slight factor in the selection of states and we can say that the neural network doesn't really consider number advantage in its selection process.

The second HLC we examined was aggressiveness, that is a state in which you most advanced pawn is $2$ or more squares further than the opponents most advanced pawn. 

\begin{figure}[]
    \centering
    \includegraphics[width=0.7\textwidth]{graphics/most_advanced_trend.png}
    \caption{Percentage of selected states containing the HLC aggressiveness}
    \label{fig:aggressiveness}
\end{figure}

From Figure \ref{fig:aggressiveness}. we can see that the aggressiveness HLC is a growing factor over as the nerual network is trained. Generally when your opponent is not skilled this strategy is considered a good one.

The last HLC that we examined was unity, the unity HLC represents the absolute average distance of your pawns from the center row of your pawns. This HLC is calculated as the row of your furthest pawn from the starting row $r_{far}$, the row of your nearest pawn from the starting row $r_{near}$. Finding the middle row is then $\frac{r_{far} + r_{near}}{2}$, we then take the absolute of the average distance from the pawns to that row. To find the point at which this value relates to a state with the HLC unity, we sampled a myriad of states and decided on the value of $0.35$. The value of $0.35$ generally allows your states to have two rows that have the majority of the pawns and one or two pawns one row away from the group.

\begin{figure}[]
    \centering
    \includegraphics[width=0.7\textwidth]{graphics/unity_trend.png}
    \caption{Percentage of selected states containing the HLC aggressiveness}
    \label{fig:unity}
\end{figure}

The literature of breakthrough implies that being patient and waiting until your opponent makes a mistake is generally the preferred strategy for winning, and the concept activation vector for Unity is generally triggered, and increases with the amount of generations the neural network is trained for. These results imply that the neural network does infact recognize successful strategies, and focuses its training in the direction of the successful strategies.

%%RUM: "Results"
\chapter{Discussion}

This section discusses the results and the future work that could span from this research. Additionally, we conclude the work.

\section{Summary}

As seen in Chapter \ref{cha:results}, our trained agent does indeed quickly rise to using the higher-level concepts. The results are promising, as we see a popular concepts like Lorentz-Horey rise in emphasis for the neural network, and a simple but effective concept like Material Advantage rise slowly but not exessively. We find the Unity concept to be disappointing as stated earlier in this paper maintaining closeness of your pawns tends to be preferential. And lastly, for a the concept Aggressiveness, the fact that it falls off early is what we expected. It would have been the preferred result if we saw a greater variability in the concepts, but this could be a result from not enough training iterations, or not a complex enough model. Importantly, these concepts are only a few possibilities of an infinite set of concepts, and the neural network could be learning a completely different concept then those that we tested. However, that was not the focus of this research, it was to examine if a neural network does move towards HLC's that we humans use in games.

\section{Future Work}

To iterate on this research, training a neural network with greater computing power will allow the researchers to hopefully achieve a super-human neural network in Breakthrough. This would give a greater confidence in the learned concepts, possibly allowing for pedagological research on the topic regarding which concepts are optimal to train people in the game environment. This could also arise from a larger neural network architecture.

An avenue of research would be to use a trained AlphaZero model that plays chess to a super human ability, and be able to draw from a more diverse set of concepts. Furthermore, one could examine the play-style of the super-human model to extract concepts in order to construct new and improved heuristics for a non-neural network based agent.

Additionally applying this method to a greater set of environment would be an interesting field of research. We envision a self driving car agent being examined with respect to a higher level concept of aggressiveness, or a mortgage agent being examined with respect to a higher level concept of gender bias.

\section{Conclusion}

The work done in this paper is only a first step in examining working agents in active environments with respect to human level concepts. If improved could have a great impact in our trust on neural network that improve our daily lives. While the testbed for the research was a discrete game environment, there is a clear way forward to a dynamic system with discrete human level concepts. Our results show that for an agent that clearly doesn't achieve human level performance it's actions are often selected with respect to human level concepts, up to 60\% of the actions.
%%RUM: "Discussion"

%% ---------------------------------------------------------------
\printbibliography{} %%RUM: "References"

%% If appendices are needed, uncomment the following line
%% and include the appendices in separate files
\appendix{}%%RUM: "Appendicies (as appropriate)

%\backmatter{} % Sections after this don't get numbers
%% We prefer that all elements be numbered

%%%%%%%%%%%%% SHOW INDEX %%%%%%%%%%%%%%%%%%
%% Index, optional.  A good idea on longer documents

% You can put instructions at the beginning of the index:
%\renewcommand{\preindexhook}{%
%  The first page number is usually, but not always,
%  the primary reference to the indexed topic.\vskip\onelineskip}

%% You may have to run "makeindex <FILENAME>" to have it be generated
%% Depending upon which package you chose.
%%
\clearforchapter{}
\printindex{}%%RUM: Not mentioned

%\backcover{}%%RUM: "Back cover (only Phd)
\end{document}

%% ---------------------------------------------------------------

%%% Local Variables:
%%% mode: latex
%%% TeX-master: t
%%% TeX-engine: xetex
%%% End:
