\chapter{Introduction\label{cha:introduction}}
%% \ifdraft only shows the text in the first argument if you are in draft mode.
%% These directions will disappear in other modes.
\ifdraft{State the objectives of the exercise. Ask yourself:
  \underline{Why} did I design/create the item? What did I aim to
  achieve? What is the problem I am trying to solve?  How is my
  solution interesting or novel?}{}

In the world of deep learning is exciting, we have models that are able to examine
images and very reliably be able to identify it's content. Deep learning models have
beaten Opthalmologists in identifing diabetic retinopathy, they've identified
cancer cells where others have not. Deep learning models have even predicted
the likelyhood a person will die in the following year with good accuracy (CITE).
These models have done these things extremely accurately, cheap, and fast.

The field of examining deep learning models to gain a richer understanding
in how they work, and what makes them so much better than humans at various tasks
is still new. This is the field of Explainable Artificial Intelligence (XAI).
If we wish to continue using artificial intelligence (AI) and machine learning (ML),
to improve our lives we must examine how they work, this is not only to improve
ourselves but these explanations are a legal requirement now in many areas, namely, legal, and
medical. Recently, XAI has generally been focused on Model Interpretability, in
particular focus on gaining insights on the concepts learned by Deep Neural Networks.

In this we thesis takes a look at how we can examine an artificial neural network (ANN)
to understand which higher level concepts (HLC) it deems important for a given state
within games. These games can be simple like Tic-Tac-Toe or Breakthrough and extremely
complicated like Chess or Go. Some of the research questions are what if an ANN values a HLC greatly
which we as humans don't find particularily important, and yet the ANN is able to defeat
us what does that mean?

The method of examining HLC's within games has generally been done by examining the
current state of the game by evaluating them with respect to a heuristic. A heuristic within
games are a evaluation of the end cost for a state given just the state, for example,
the amount of pieces left within a game of chess. Intuitively, the amount of pieces
left is a good estimate for a state in chess, this is a higher level concept we
use to evaluate a chess position. The piece amount can be considered as a lower-level concept
than other concepts we use. For instance, grand master chess players evaluate a position w.r.t. states
where the king is safe from attack, or the structure of the pawn positions.
The idea is to examine a neural networks evaluation of a state regarding
those higher-level concepts, if human players evaluate the king safety of a state
low but the neural network highly values it, and the neural network plays better
than the player. There could be an avenue for us to improve our game by considering king safety more thoroughly.

In this thesis we take an example game of Breakthrough, train a neural network to play the game
using only self-play. That is, the neural network is trained only by playing against
itself with no outside input other than the rules of the game itself. And we examine
the neural networks against popular Higher-level concepts (heuristics) that we use
to play the game.

The neural network architecture is based of the popular AlphaZero neural network
used to play Chess, Shogi, and Go on a superhuman level. (REMOVE ME) Unfortunately due to lack
of thousands of TPU's this research can't go as far as to examine those more complicated
games. (REMOVE ME)


% The chess engine DeepBlue was the first chess engine to win a chess match agains a
% chess grand master. It's power in the world of chess was a marvel. DeepBlue used tree-
% search to evaluate it's position and possible moves, doing an immense amount of calcluations
% in order to arrive at the best move it could given the time constraints. The tree search
% approach is dependent on knowing which board state is good and what board state is bad,
% This can be easy when the boardstate is such that you're guaranteed to have been checkmated
% in one move, and it's incredibly hard when you boardstate is such that both you and your
% opponent have 16 pieces and are only a handfull of moves in. Generally the boardstate
% is a function of the pieces you have, how safe they are, and how opportunistic you are to
% capture your opponents pieces. These values are not obvious and require a domain expert
% to describe the situations and give the values for these cases.

% The process of evaluating chess positions did not change for many years. With the revival
% of deep neural networks this changed. A neural network is a network of connected nodes (neurons)
% that each how a specific weight. These networks are fed input which is passed through the
% network and it is updated by the weights of the neurons, the input is morphed into
% a more comprehensible value and these values are then used to generate some output.
% The output can be a binary classification for instance input=Image output=\{cat, dog\}.
% Processing the pixels and altering them to numerical values to find the output cat or dog
% is not intuitive and understanding of the values of the neurons in the network
% is not considered fruitful.



% This approach of
% chess engines is still popular today but not the reigning champion anymore. In todays
% best chess engines deep neural networks are used to

% Deep neural network machine learning models have an amazing impact on modern society today,
% we use them to know which items to purchase, to know which people we should love, and we
% use them to drive us to work. These tasks given to deep learning models are all niceties
% for us to enjoy. However, as our dependance on them increases and their astounding success
% in any and all tasks continues, we will want to utilize them for all things. This leads
% to a utopia where decision making is removed from the hands of humans. The question we
% must ask ourselves is when do we want to know why a decision was taken, is it only when
% the decision taken is not preferrable to us. For instance, if a deep neural network model
% used by your lending authority decides you are not worty of a loan. These questions should
% have answers but as it stands today no such answer is possible.

% We can use methods like saliency maps to map a deep learning model emphasis on pixels to
% hopefully understand what portions the model is focusing on. This method however, tells us
% nothing about what

% results that are nothing but amazing.
